%%%%%%%%%%%%%%%%%%%%%%%%%%%%%
%% Math homework template %%%
%% ---------------------- %%%
%%   Author: Stefan Eng   %%%
%%%%%%%%%%%%%%%%%%%%%%%%%%%%%


\documentclass{article}

\usepackage[utf8]{inputenc}
\usepackage[top=1.5in,left=1in,right=1in,bottom=1.5in,headheight=1in]{geometry}
\usepackage{fancyhdr}
\usepackage{lastpage}
\usepackage{amsmath}
\usepackage{enumerate}


%%%%%%%%%%%%%%%%%%%%%%%%%%%%%
%%% Information for Title %%%
%%%%%%%%%%%%%%%%%%%%%%%%%%%%%

%%% Fill this in with your own information

\newcommand{\name}{
% Enter your name here
Stefan Eng
}

\newcommand{\professor}{
% Professors name goes here
William Watkins
}

\newcommand{\classcode}{
% Class name "code" goes here. e.g, Math320
Math 340
}

\newcommand{\classname}{
% Class name goes here
Introduction to Probability
}

\newcommand{\duedate}{
% Due date goes here
9/9/13
}

\newcommand{\chSec}{
% Enter in chapter sections
2
}
    
\newcommand{\problems}{
% Enter in the problem numbers we are working on
29, 37, 55, 59, 71, 74
}

\newcommand{\assignment}{
% Enter assignment number
2
}

%%%%%%%%%%%%%%%%%%%%%%%%%%%%%


%%% Heading -- No need to edit %%%
\pagestyle{fancy}
\rhead{\name \\ \professor \\ \classcode \\ \duedate}
\lhead{Chapter:\chSec\ \\ Problems:\problems}
\cfoot{Page\ \thepage\ of\ \pageref{LastPage}}
%%%



% Good reference for math commands web.ift.uib.no/Teori/KURS/WRK/TeX/symALL.html

\begin{document}

%%% Make the title %%%
\begin{center}
\textsc{\Large \classname}\\[.3cm]
\textsc{\Large Homework \assignment}
\end{center}
%%% End title %%%

%%% Start Assignment Here %%%
\section*{Problem 29}
    \begin{enumerate}[a)]
    
        \item The experiment is randomly selecting two jurors from a group of two women and four men. A sample point is selecting one of the men and one of the women.
        
        \item Given that we have $W_1, W_2, M_1, M_2, M_3, M_4$ the sample space is:
            
            %%% Sample space
            \begin{tabular}{c c c c c c}
                $E_1 = \{W_1, W_2\},$ & $E_6 = \{W_2, M_1\},$ & $E_{10} = \{M_1, M_2\},$ & $E_{13} = \{M_2, M_3\},$ & $E_{15} = \{M_3, M_4\}.$ \\                              
                $E_2 = \{W_1, M_1\},$ & $E_7 = \{W_2, M_2\},$ & $E_{11} = \{M_1, M_3\},$ & $E_{14} = \{M_2, M_4\},$ & \\
                $E_3 = \{W_1, M_2\},$ & $E_8 = \{W_2, M_3\},$ & $E_{12} = \{M_1, M_4\},$ &                       &\\        
                $E_4 = \{W_1, M_3\},$ & $E_9 = \{W_2, M_4\},$ &                       &                       & \\
                $E_5 = \{W_1, M_4\},$ &                       &                       &                       &
            \end{tabular}
            
        \item The probability that both of the jurors are women is: $P(E_1) = 1/15$
            
    \end{enumerate}
    
\section*{Problem 37}
    \begin{enumerate}[a)]
        \item Since order does matter, the business women can travel in $6 * 5 * 4 * 3 * 2 = 720$ ways.
        \item The woman either travels to Denver first or San Francisco first. Since we only have two outcomes the probability is $.5$.
    \end{enumerate}

\section*{Problem 55}
    \begin{enumerate}[a)]
        \item Since the order the nurses getting picked does not matter, we can use combinations: $${90 \choose 10} = \frac{90!}{10!\ 80!} = 2.07 \times 10^{18}$$
        \item To get the number of times that 4 male nurses and 6 female nurses are seen in the sample points, we first need to choose 4 male nurses out of 20. Then choose 6 female nurses out of the remaining 70 nurses. The total number of combinations of nurses is given by:
            $$P(A) = \frac{n_a}{N} = \frac{{20 \choose 4}{70 \choose 6}}{{90 \choose 10}} $$
    \end{enumerate}

\section*{Problem 59}
    \begin{enumerate}[a)]
        \item There are ${52 \choose 5}$ sample points. Since there are 4 of each kind of card in the deck there is $4^5$ ways to draw the straight we want. Therefore the probability of drawing the straight we want is:
        $$P(A) = \frac{n_a}{N} = \frac{1024}{{52 \choose 5}} = 3.94 \times 10^{-4}$$
        \item Using the result of part a, with the knowledge that there are 10 other ways to rearrange the straight, we get:
        $$P(A) = 3.94 \times 10^{-3}$$
    \end{enumerate}
    
\section*{Problem 71}
    Given that, $P(A) = .5, P(B) = .3 P(A \cap B) = .1$
    \begin{enumerate}[a)]
        \item $P(A|B) = \frac{P(A \cap B)}{P(B)} = \frac{.1}{.3} = \frac{1}{3}$
        \item $P(B|A) = \frac{P(B \cap A)}{P(A)} = \frac{.1}{.5} = \frac{1}{5}$
        \item \begin{align*}
                P(A|A \cup B) &= \frac{P(A \cap (A \cup B))}{P(A \cap B)}\\[.2cm]
                              &= \frac{P(A \cup (A \cap B))}{P(A) + P(B) - P(A \cap B)}\\[.2cm]
                              &= \frac{P(A) + P(A \cap B) - P(A \cap (A \cap B))}{P(A) + P(B) - P(A \cap B)}\\[.2cm]
                              &= \frac{.5}{.5 + .3 - .1}\\[.2cm]
                              &= \frac{5}{7}
              \end{align*}
        \item $P(A|A \cap B) = \frac{P(A \cap (A \cap B)}{P(A \cap B} = 1$
        \item \begin{align*}
                P(A \cap B | A \cup B) &= \frac{P((A \cap B) \cap (A \cup B))}{.7}\\[.2cm]
                                       &= \frac{P(A \cap B)}{.7}\\
                                       &= \frac{1}{7}
              \end{align*}
    \end{enumerate}
    
\section*{Problem 74}
    \


\end{document}

%%% End assignment %%%
