%%% Math 340
%%% Homework 3

\documentclass{article}

\usepackage[utf8]{inputenc}
\usepackage[top=1.5in,left=1in,right=1in,bottom=1.5in,headheight=1in]{geometry}
\usepackage{fancyhdr}
\usepackage{lastpage}
\usepackage{amsmath,amsthm}
\usepackage{enumerate}

\newtheoremstyle{problem}
{10pt}% Above space
{10pt}% Below space
{}%         Body font
{}%         Indent amount (empty = no indent, \parindent = para indent)
{\bfseries}% Thm head font
{}%        Punctuation after thm head
{\newline}%     Space after thm head (\newline = linebreak)
{\thmname{#1}\thmnote{ #3}}%         Thm head spec

\theoremstyle{problem}
\newtheorem{prob}{Problem}

%%% Heading -- No need to edit %%%
\pagestyle{fancy}
\rhead{
  Stefan Eng \\ 
  William Watkins \\ 
  Math 340 \\ 
  9/16/13
}
\lhead{
  Sections:2.7-2.10 \\ 
  Problems: 77a,c,h, 95, 129, 133, 137
}
\cfoot{Page\ \thepage\ of\ \pageref{LastPage}}
%%% 



% Good reference for math commands web.ift.uib.no/Teori/KURS/WRK/TeX/symALL.html

\begin{document}

%%% Make the title %%%
\begin{center}
  \textsc{\Large Introduction to Probability}\\[.3cm]
  \textsc{\Large Homework 3}
\end{center}
%%% End title %%%

%%% Start Assignment Here %%%
\begin{prob}[77a]
$P(A) = \mathbf{.40}$
\end{prob}
\begin{prob}[77c]
$P(A \cap B) = \mathbf{.10}$
\end{prob}
\begin{prob}[77h]
$P(A|B) = \frac{P(A \cap B)}{P(B)} = \frac{.10}{.37} = \mathbf{\frac{10}{37}}$
\end{prob}


\begin{prob}[95]
Two events A and B are such that $P(A) = .2,$ $P(B) = .3,$ and $P(A \cup B) = .4$
\begin{enumerate}[a)]
\item $P(A \cap B) = P(A) + P(B) - P(A \cup B) = .2 + .3 - .4 = \mathbf{.1}$
\item $P(\overline{A} \cup \overline{B}) = P(\overline{A \cap B}) = 1 - P(A \cap B) = \mathbf{.9}$
\item $P(\overline{A} \cap \overline{B}) = P(\overline{A \cup B}) = 1 - P(A \cup B) = \mathbf{.6}$
\item $P(\overline{A}|B) = \frac{P(\overline{A} \cap B)}{P(B)} = \frac{P(B) - P(A \cap B)}{P(B)} = \frac{2}{3}$
\end{enumerate}
\end{prob}

\begin{prob}[129]
Given:
\begin{tabular}{l c r}
$P(+|F) = .7$ & $P(+|M) = .4$ & $P(M) = .25$\\
$P(-|F) = .3$ & $P(-|M) = .6$ & $P(F) = .75$
\end{tabular}\\

\noindent
Using Bayes Theorem,
\begin{align*}
P(M|-) &= \frac{P(-|M)P(M)}{P(-|M)P(M) + P(-|F)P(F)}\\
       &= \frac{.6 \times .25}{.6 \times .25 + .3 \times .75}\\
       &= .4
\end{align*}
\end{prob}

\begin{prob}[133]
Let
\begin{tabular}{l l}
K = Knows the answer to the question & R = Gets the right answer \\ 
G = Guesses the answer               & W = Gets the wrong answer
\end{tabular}\\[.3cm]
Then
\begin{tabular}{c c c}
$P(K) = .8$ & $P(R|G) = .25$ & $P(R|K) = 1$\\
$P(G) = .2$ & $P(W|G) = .75$ & $P(W|K) = 0$
\end{tabular}\\[.3cm]
The probability that the student really knew the answer given that they got it correct is:
\begin{align*}
P(K|R) &= \frac{P(R|K)P(K)}{P(R|K)P(K) + P(R|G)P(G)}\\
       &= \frac{.8}{.8 + .25 \times .2}\\
       &\approx .9412
\end{align*}
\end{prob}

\begin{prob}[137]

\end{prob}


\end{document}

%%% End assignment %%%
