%%% Lecture Notes for Math 340
%%% Author: Stefan Eng
\documentclass{article}

\usepackage[utf8]{inputenc}
\usepackage[top=1.5in,left=1in,right=1in,bottom=1.5in,headheight=1in]{geometry}
\usepackage{fancyhdr}
\usepackage{amsmath,amsthm}
\usepackage{lastpage}
\usepackage{multicol}

\newtheorem*{nota}{Notation}
\newtheorem*{q}{Question}
\newtheorem*{ans}{Answer}
\newtheorem{defn}{Definition}
\newtheorem{ex}{Example}

%%%%%%%%%%%%%%%%%%%%%%%%%%%%%
%%% Information for Title %%%
%%%%%%%%%%%%%%%%%%%%%%%%%%%%%

%%% Fill this in with your own information

\newcommand{\name}{
% Enter your name here
Stefan Eng
}

\newcommand{\classcode}{
% Class name "code" goes here. e.g, Math320
Math 340
}

\newcommand{\classname}{
% Class name goes here
Introduction to Probability
}


%%%%%%%%%%%%%%%%%%%%%%%%%%%%%


%%% Heading -- No need to edit %%%
\pagestyle{fancy}
\rhead{\name \\ \classcode}
\cfoot{Page\ \thepage\ of\ \pageref{LastPage}}
%%%

\begin{document}
%%% Make the title %%%
\begin{center}
\textsc{\Large \classname}\\[.3cm]
\textsc{\Large Lecture Notes - Conditional Probability}\\[.3cm]
\textsc{\large September, 5th \& 9th 2013}
\end{center}
%%% End title %%%


\section*{Intuition}
Here we will build up some intuition for what conditional probability means.
\begin{ex}
Roll a ``loaded'' die\\
\begin{tabular}{r | c c c c c c}
      & 1   & 2  & 3   & 4   & 5   & 6 \\
\hline
Prob. & .05 & .2 & .15 & .25 & .05 & .3
\end{tabular}\\[.3cm]

Event \textbf{A}: Odd number is rolled.

\begin{equation} \label{PofA}
  P(A) = P(\{1,3,5\}) = P(\{1\}) + P(\{3\}) + P(\{5\}) = .05 + .15 + .05 = .25
\end{equation}

Event \textbf{B}: Roll a 3.

\begin{equation} \label{PofB}
  P(B) = P(\{3\}) = .15
\end{equation}

\begin{q}What is the probability of 3 given that the die is odd?\end{q}

\begin{nota}
  $P(B|A)$: What is the probability of B given A already occurred?
\end{nota}

Now our sample space has been restricted. The only possible options that the die could be are 1, 3 or 5. So lets make another table with the sample space restricted:

\begin{tabular}{r | c c c}
      & 1   & 3   & 5 \\
\hline
Prob. & .05 & .15 & .05 
\end{tabular}\\[.3cm]

This does not appear right. According to our axioms, $P(S) = 1$, (where S is the sample space) and we can see in our table that $P(S) = .2$. So how are we going to ``normalize'' the sample space? If we divide each number by the sum of the new sample space as seen below:


\begin{align*} \label{cond}
              P(S)                           &= 1 \\
              P(\{1,2,3,4,5,6\}              &= 1 \\
              P(odd) = P(\{1,3,5\})          &= 1 - P(\{2,4,6\}) \\
              P(\{1\}) + P(\{3\}) + P(\{5\}) &= 0.25 \\
              \frac{P(\{1\})}{0.25} + \frac{P(\{3\})}{0.25} + \frac{P(\{5\})}{0.25} &= 1
\end{align*}

\begin{ans} 
So now it is easy to see the $P(B|A)$ or the probability of 3 given that the die is odd is: $$P(B|A) = \frac{0.15}{0.25} = 0.6$$
\end{ans}


\end{ex}
\end{document}

%%% End assignment %%%

