%%% Lecture Notes for Math 340
%%% Author: Stefan Eng
\documentclass{article}

\usepackage[utf8]{inputenc}
\usepackage[top=1.5in,left=1in,right=1in,bottom=1.5in,headheight=1in]{geometry}
\usepackage{fancyhdr}
\usepackage{lastpage}
\usepackage{amsmath,amsthm,amssymb}

% mymacros.sty
\usepackage{math340_macros}

%%%%%%%%%%%%%%%%%%%%%%%%%%%%%

\newtheorem{axiom}{Axiom}

%%% Heading %%%
\pagestyle{fancy}
\rhead{Stefan Eng}
\cfoot{Page\ \thepage\ of\ \pageref{LastPage}}
%%%

\begin{document}

% Uncomment when we want title page and table of contents
% % Title page for the Math 340 class notes

\begin{titlepage}
\thispagestyle{empty}
\begin{center}
\textsc{\LARGE \bfseries Introduction to Probability}\\[.3cm]
\HRule \\[.5cm]
\textsc{\large California State University\\Northridge}\\
\HRule \\[1cm]
\textsc{\large Stefan Eng}
\vfill

{\large Fall 2013}

\end{center}
\end{titlepage}

% \tableofcontents
% \vfill
% \pagebreak

\section{Discrete Probability}

\subsection{Terms}
Some basic definitions.

\subsection{Set Theory}
Some basic set theory is needed in the study of probability.

\subsection{Axioms}
$E$ is an event. $\Omega$ is the sample space.
\begin{axiom} \label{axiom1}
The probability of an event is a non-negative real number
$$
P(E) \in \mathbb{R}, P(E) \geq 0
$$
\end{axiom}

\begin{axiom} \label{axiom2}
$$
P(\Omega) = 1
$$
\end{axiom}

\begin{axiom} \label{axiom3}
For countable, disjoint events $E_1,E_2,\ldots$:
$$ 
P(E_1 \cup E_2 \cup \cdots) = \displaystyle \sum_{i=1}^{\infty} P(E_i)
$$
\end{axiom}

\subsection{Calculating Probability}


\end{document}

%%% End assignment %%%

