%%% Math 340
%%% Homework 5

\documentclass{article}

\usepackage[utf8]{inputenc}
\usepackage[top=1.5in,left=1in,right=1in,bottom=1.5in,headheight=1in]{geometry}
\usepackage{fancyhdr}
\usepackage{lastpage}
\usepackage{amsmath,amsthm}
\usepackage{enumerate}
\usepackage{listing}

\newtheoremstyle{problem}
{\topsep}% Above space
{\topsep}% Below space
{}%         Body font
{}%         Indent amount (empty = no indent, \parindent = para indent)
{\bfseries}% Thm head font
{\vspace{5pt}}%        Punctuation after thm head
{\newline}%     Space after thm head (\newline = linebreak)
{\thmname{#1}\thmnote{ #3}\\}%         Thm head spec

\theoremstyle{problem}
\newtheorem{prob}{Problem}

%%% Heading -- No need to edit %%%
\pagestyle{fancy}
\rhead{
  Stefan Eng \\ 
  William Watkins \\ 
  Math 340 \\ 
  9/30/13
}
\lhead{
  Chapter 3\\
  39, 41, 61, 67, 73
}
\cfoot{Page\ \thepage\ of\ \pageref{LastPage}}
%%% 

% No indent for whole page
\setlength\parindent{0pt}

\begin{document}

%%% Make the title %%%
\begin{center}
  \textsc{\Large Introduction to Probability}\\[.3cm]
  \textsc{\Large Homework 5}
\end{center}
%%% End title %%%

%%% Start Assignment Here %%%
\begin{prob}[39]
  Given that blah blah..
  \begin{enumerate}[a)]
  \item Exactly two of the four components last longer than 1000 hours
    \begin{align*}
      p(2) &= {4 \choose 2} (.2^2) (.8^2)\\
      &= .1536
    \end{align*}
  \item The subsystem operates longer than 1000 hours
    \begin{align*}
      p(\text{longer than 1000 hours}) &= 1 - p(\text{less than 1000 hours})\\
      &= 1 - p(3)\\
      &= 1 - {4 \choose 3} (.2^3) (.8)\\
      &= .9744
    \end{align*}
  \end{enumerate}
\end{prob}
%
\begin{prob}[41]
To get at least 10 questions right on the test we must have, $p(y >= 10)$. This can be represented as
\begin{align*}
\displaystyle \sum_{x = 10}^{15} &= {15 \choose x} \left(\frac{1}{5}\right)^x \left(\frac{4}{5}\right)^{15-x}\\
&= .0001
\end{align*}

\end{prob}
%
\begin{prob}[61]
  80\% of people donating blood have the Rhesus(Rh) factor present.
  \begin{enumerate}[a)]
  \item With five people selected, what is the probability that at least one does not have the Rh factor?
    \begin{align*}
      p(\text{at least one does not have Rh}) &= 1 - p(\text{everyone has Rh})\\
      &= 1 - (.8)^5\\
      &= .6723
    \end{align*}
  \item With five people selected, what is the probability that at most four have the Rh factor?
    \begin{align*}
      p(\text{at most four have the Rh factor}) = &= 1 - p(\text{everyone has Rh})\\
      &= .6723
    \end{align*}
  \item Using the equation:
    \begin{align*}
    {n \choose 5} (.8)^5 (.2)^{n-5} &= p(y \geq 5)\\
    &= .90
  \end{align*}  
  Testing each value of $n$, the first case where $p(y \geq 5) > .90$ is when there are \textbf{8} people selected.
  \end{enumerate}
\end{prob}
%
\begin{prob}[67] Finding a success on 5th try:
  \begin{align*}
    p(5) &= (.7)^{4}(.3)\\
    &= .072
  \end{align*}
\end{prob}
%
\begin{prob}[73]
  Probability of an error in the audit is \textbf{.9}
  \begin{enumerate}[a)]
  \item Error is found on the third company:
    \begin{align*}
      p(3) &= (.1)^2 (.9)\\
      &= .009
    \end{align*}
  \item Error is found on or after the third audit:
    \begin{align*}
      p(y \geq 3) &= 1 - p(y < 3)\\
      &= 1 - [p(1) + p(2)]\\
      &= 1 - [.9 + (.01)(.90)]\\
      &= 1 - .99\\
      &= .01
    \end{align*}
  \end{enumerate}
\end{prob}
%
\end{document}

%%% End assignment %%%
