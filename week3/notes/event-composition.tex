%%% Lecture Notes for Math 340
%%% Author: Stefan Eng
\documentclass{article}

\usepackage[utf8]{inputenc}
\usepackage[top=1.5in,left=1in,right=1in,bottom=1.5in,headheight=1in]{geometry}
\usepackage{fancyhdr}
\usepackage{amsmath,amsthm}
\usepackage{lastpage}
\usepackage{multicol}

%% Theorem stuff
\theoremstyle{theorem}
\newtheorem{thm}{Theorem}


\theoremstyle{remark}
\newtheorem*{rmk}{Remark}

\renewcommand\qedsymbol{}

%%%%%%%%%%%%%%%%%%%%%%%%%%%%%
%%% Information for Title %%%
%%%%%%%%%%%%%%%%%%%%%%%%%%%%%

%%% Fill this in with your own information

\newcommand{\name}{
% Enter your name here
Stefan Eng
}

\newcommand{\classcode}{
% Class name "code" goes here. e.g, Math320
Math 340
}

\newcommand{\classname}{
% Class name goes here
Introduction to Probability
}


%%%%%%%%%%%%%%%%%%%%%%%%%%%%%


%%% Heading -- No need to edit %%%
\pagestyle{fancy}
\rhead{\name \\ \classcode}
\cfoot{Page\ \thepage\ of\ \pageref{LastPage}}
%%%

\begin{document}
%%% Make the title %%%
\begin{center}
\textsc{\Large \classname}\\[.3cm]
\textsc{\Large Lecture Notes - Event Composition}\\[.3cm]
\textsc{\large September 9th \& 11th, 2013}\\[1cm]
\end{center}
%%% End title %%%

\begin{thm}[The Multiplicative Law of Probability]
  The probability of the intersection of two events A and B is
  \begin{align}
    P(A \cap B) &= P(A) P(B|A)\\
                &= P(B) P(A|B)
  \end{align}
  If A and B are independent, then
  $$
  P(A \cap B) = P(A) P(B)
  $$
\end{thm}
\begin{proof}
This follows directly from the definition of conditional probability:
\begin{equation} \label{cond}
  P(A|B) = \frac{P(A \cap B)}{P(B)}
\end{equation}
\end{proof}
%%%%%%%%%%%%%%%%%%%%%%%%%%%%%%%%%%%%%%%%%%%%%%%%%%%%%%%%%%%%%%%%%%%%%
\begin{thm}[The Additive Law of Probability]
  The probability of the union of two events A and B is
  \begin{equation} \label{additive}
    P(A \cup B) = P(A) + P(B) - P(A \cap B)
  \end{equation}
  If A and B are mutually exclusive events, $P(A \cap B) = 0$ and
  \begin{equation} \label{additive-excl}
    P(A \cup B) = P(A) + P(B)
  \end{equation}
  but equation \ref{additive} works every time.
\end{thm}

\begin{proof}
Insert a proof here
\end{proof}

%%%%%%%%%%%%%%%%%%%%%%%%%%%%%%%%%%%%%%%%%%%%%%%%%%%%%%%%%%%%%%%%%%%%%
\begin{thm}
  If A is an event, then
  \begin{equation}
    P(A) = 1 - P(\overline{A})
  \end{equation}
\end{thm}
\begin{proof}
Observe that $S = A \cup \overline{A}$. Because $A$ and $\overline{A}$ are mutually exclusive events, $P(A \cap \overline{A}) = 0$, it follows that $P(S) = P(A) + P(\overline{A})$. Therefore, $P(A) + P(\overline{A}) = 1$ and the results follows.
\end{proof}

\begin{rmk}
This is a useful tool when $P(\overline{A})$ is easier to calculate than $P(A)$
\end{rmk}
%%%%%%%%%%%%%%%%%%%%%%%%%%%%%%%%%%%%%%%%%%%%%%%%%%%%%%%%%%%%%%%%%%%%%


\end{document}


%%% End assignment %%%

