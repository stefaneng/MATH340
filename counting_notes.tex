%%% Lecture Notes for Math 340
%%% Author: Stefan Eng
\documentclass{article}

\usepackage[utf8]{inputenc}
\usepackage[top=1.5in,left=1in,right=1in,bottom=1.5in,headheight=1in]{geometry}
\usepackage{fancyhdr}
\usepackage{amsmath,amsthm}
\usepackage{lastpage}
\usepackage{multicol}

\newtheorem{defn}{Definition}
\newtheorem{ex}{Example}[defn]

%%%%%%%%%%%%%%%%%%%%%%%%%%%%%
%%% Information for Title %%%
%%%%%%%%%%%%%%%%%%%%%%%%%%%%%

%%% Fill this in with your own information

\newcommand{\name}{
% Enter your name here
Stefan Eng
}

\newcommand{\classcode}{
% Class name "code" goes here. e.g, Math320
Math 340
}

\newcommand{\classname}{
% Class name goes here
Introduction to Probability
}


%%%%%%%%%%%%%%%%%%%%%%%%%%%%%


%%% Heading -- No need to edit %%%
\pagestyle{fancy}
\rhead{\name \\ \classcode}
\cfoot{Page\ \thepage\ of\ \pageref{LastPage}}
%%%



% Good reference for math commands web.ift.uib.no/Teori/KURS/WRK/TeX/symALL.html

% \newcommand{\perm}{{P_{\hspace{3pt}k}}^{\hspace{-5.25pt}n}}

\newcommand{\perm}[3]{
#1_{\hspace{2pt}#3}^{\hspace{1pt}#2}
}


\begin{document}
%%% Make the title %%%
\begin{center}
\textsc{\Large \classname}\\[.3cm]
\textsc{\Large Lecture Notes - Counting}\\[.3cm]
\textsc{\large September, 4 2013}
\end{center}
%%% End title %%%

\section*{Permutations}
\begin{defn}
A \textbf{permutation} is an ordered arrangement of \textbf{k} objects from \textbf{n} objects.
\end{defn}
\begin{equation}
\perm{P}{n}{k} = P(n,k) = \frac{n!}{(n-k)!} = n (n-1)\ldots(n-k+1)
\end{equation}
\begin{ex}
There is a class of 30 students and there are 10 front row seats. How many ways are there to seat 10 of the 30 in the front row?\\
\indent \textit{Answer:} $$ \perm{P}{30}{10} = \frac{30!}{20!} = 30 * 29 * \ldots * 21 $$
\end{ex}

\section*{Combinations}
\begin{defn}
A \textbf{combination} is an unordered subset of \textbf{k} objects from \textbf{n} objects. Said as, \textbf{n choose k}.
\end{defn}

\begin{equation}
\perm{C}{n}{k} = {n \choose k} = \frac{n!}{k!\ (n-k)!}
\end{equation}

\begin{ex}
In Professor Watkins's Math 103 class, 12 students wanted to add. He could only take 4 of them.\\
\indent \textit{Answer:} $${12 \choose 4} = \frac{12*11*10*9}{4*3*2} = 495 $$
\end{ex}

\begin{defn}
The \textbf{binomial theorem} is a use of combinations.
\end{defn}

\begin{equation}
(x+y)^n=\sum_{k=0}^n\binom nk x^{n-k}y^k
\end{equation}
\section*{Partition}
\begin{defn}
A \textbf{partition} is when a set of \textbf{n} object is divided \textbf{k} distinct, nonoverlapping groups containing $n_1,n_2, \ldots , n_k$ objects respectively. The following condition must hold:
\end{defn}
$${\displaystyle\sum\limits_{i=0}^k n_i}= n_1 + n_2 + \ldots + n_k = n$$
Then,
\begin{equation}
{n \choose n_1,n_2,\ldots,n_k} = \frac{n!}{n_1!\ n_2!\ \ldots\ n_k!}
\end{equation}

\newpage

\begin{defn}
The terms ${n \choose n_1,n_2,\ldots,n_k}$ are often called \textbf{multinomial coefficients} because they occur in the expansion of the \textbf{multinomial term} $x_1 + x_2 + \ldots + x_k$ raised to the nth power.
\end{defn}

\begin{equation}
(x_1 + x_2  + \cdots + x_m)^n = \sum_{k_1+k_2+\cdots+k_m=n} {n \choose k_1, k_2, \ldots, k_m} \prod_{1\leq t\leq m}x_{t}^{k_{t}}
\end{equation}

\begin{ex}
How many 13-card bridge hands contain 4 hearts, 3 spades, 1 diamond, and 4 clubs?

\indent \textit{Answer:} 
\end{ex}

\begin{ex}
In a class of 30 students, there will be 4 A's, 5 B's, 18 C's, 2 D's, and 0 F's. How many ways are there to assign these grades.\\
\indent \textit{Answer 1 (Partitions):} 
$$n = 30, k = 5$$
$$n_1 = 4,\ n_2 = 6,\ n_3 = 18,\ n_4 = 2,\ n_5 = 0$$
$${30 \choose 4,6,18,2} = \frac{30!}{4!\ 6!\ 18!\ 2!}$$
\\
\indent \textit{Answer 2 (Combinations):}
\begin{multicols}{2}
$${30 \choose 4}\ get\ A's$$
$${26 \choose 6}\ get\ B's$$
$${20 \choose 18}\ get\ C's$$
$${2 \choose 2}\ get\ D's$$

\vfill
\columnbreak
$$
\begin{aligned}
& ={30 \choose 4} {26 \choose 6} {20 \choose 18} {2 \choose 2}\\
& =\frac{30!}{4!\ 26!}\ \frac{26!}{6!\ 20!}\ \frac{20!}{18!\ 2!}\ \frac{2!}{2!}\\
& =\frac{30!}{4!\ 6!\ 18!\ 2!}\\
\end{aligned}
$$
\end{multicols}
\end{ex}

\begin{defn}
\textbf{Pascal's Triangle}  is a triangular array of the binomial coefficients. Row n and column k have the value of ${n \choose k}$. To calculate the next row, add up the two numbers above it.
\end{defn}

\begin{multicols}{2}
  %%% Start regular Pascal's Triangle
\begin{tabular}{rccccccccc}
$n=0$: & & & & & 1 & & & & \\
$n=1$: & & & & 1 & & 1 & & & \\
$n=2$: & & & 1 & & 2 & & 1 & & \\
$n=3$: & & 1 & & 3 & & 3 & & 1 & \\
$n=4$: & 1 & & 4 & & 6 & & 4 & & 1%
\end{tabular}

%%% Row-Column Pascal's Triangle
\begin{tabular}{r|r|c|c|c|c|c|c|c}
& 0 & 1 & 2 & 3 & 4 & 5 & 6 & 7\\
\hline
0 & 1 & & & & & & &\\
\hline
1 & 1 & 1 & & & & & &\\
\hline
2 & 1 & 2 & 1 & & & & &\\
\hline
3 & 1 & 3 & 3 & 1 & & & &\\
\hline
4 & 1 & 4 & 6 & 4 & 1 & & &%
\end{tabular}

\end{multicols}

\end{document}

%%% End assignment %%%

